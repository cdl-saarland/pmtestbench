\title{Inferred Port Mapping for the AMD Zen+ Microarchitecture}
\author{Fabian Ritter, Sebastian Hack -- Compiler Design Lab, Saarland University}
\maketitle

\noindent
This port mapping was inferred with the technique described in the paper \emph{Explainable Port Mapping Inference with Sparse Performance Counters for AMD’s Zen Architectures} by Fabian Ritter and Sebastian Hack.
Limitations and caveats are also discussed there.
The columns contain the following information:
    \begin{description}[labelindent=0.5cm, leftmargin=1.0cm]
        \item[Instruction Scheme:] An instruction scheme, i.e., an instruction representation that abstracts from concrete operands. Click on one to open the corresponding entry in the ISA reference.
        \item[\(\mathit{tp}^{-1}\):] The inverse throughput we measured for the instruction scheme, i.e., the number of cycles required on average to execute instances of the instruction scheme in a steady state.
            For instance, an inverse throughput of 0.25 cycles means that 4 instructions are executed simultaneously.
        \item[\#\(\mu\)ops:] The number of micro-operations (\(\mu\)ops) observed when executing instances of the instruction scheme.
            We obtain this number by correcting the readings of a hardware performance counter as described in the paper.
            If such a correction took place for a particular instruction scheme, the addition is made explicit in the table.
            An asterisk here marks that the number of \(\mu\)ops does not match the number of characterized \(\mu\)ops in the port usage.
        \item[Port Usage:] The ports used by the \(\mu\)ops of the instruction scheme.
            For instance, the port usage \(1\times [a] + 2\times [b, c]\) means that the instruction scheme is decomposed into one \(\mu\)op that can only be executed on port \(a\) and two \(\mu\)ops that can be executed on port $b$ or port $c$.
            All \(\mu\)ops need to be executed when executing an instance of the instruction scheme.
            An empty entry means that no valid and stable port usage could be inferred in the last stage of the algorithm.
    \end{description}

The ports roughly correspond to the following hardware resources:
\begin{center}
    \begin{tabular}{clp{0.6\textwidth}}
        \toprule
        \textbf{Port} & \textbf{Resource} & \textbf{Comment}\\
        \toprule
        0 & FP0 & \multirow{4}{*}{Floating Point and Vector Units}\\
        1 & FP1 & \\
        2 & FP2 & \\
        3 & FP3 & \\
        4 & AGU & \multirow{2}{*}{Address Generation Units}\\
        5 & AGU/store & \\
        6 & ALU & \multirow{4}{*}{Scalar Integer Arithmetic and Logical Units}\\
        7 & ALU & \\
        8 & ALU & \\
        9 & ALU & \\
        \bottomrule
    \end{tabular}
\end{center}

